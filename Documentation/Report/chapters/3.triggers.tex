%---------------------------------------------------------------------
%   TRIGGERS CHAPTER
%---------------------------------------------------------------------

\chapter{Views and triggers}
This section will address the behaviour of the triggers developed for the project. Most of them were developed with
the aim of operating on the materialized view tables, as per the specification of the project, but some include functionalities
that we chose to implement via triggers, such as the handling of the insolvent flag, the failed payments and audits mechanism
and the service activation schedule creation.

\section{SQL View Code}
Since most of the triggers are related to the materialized views required for the sales report, this section prefaces
the actual implementation with the code that would have been used for the actual views, had they not been materialized.
Most of the code can still be found reused in the actual triggers, and the views share their name with the materialized
view table so that they are easier to trace.

\subsection{Number of total purchases per package}

\begin{lstlisting}[style=SQL]
CREATE VIEW PurchasesPerPackage AS
SELECT Name, COUNT(Pkg_ID) AS Purchases
FROM Orders INNER JOIN Service_Pkgs ON Orders.Pkg_ID=Service_Pkgs.ID
GROUP BY Pkg_ID;
\end{lstlisting}

In the triggers, it is mapped by \verb NewPackage and \verb OrderCompleted , which respectively create an entry in the
\verb PurchasesPerPackage table with zero purchases and increment it by one for each successful order containing the package.

\subsection{Number of total purchases per package and validity period}

\begin{lstlisting}[style=SQL]
CREATE VIEW PurchasePerPackagePeriod AS
SELECT  Name,
        Validity_Periods.Months AS Months,
        COUNT(Orders.Pkg_ID) AS Purchases
FROM Orders
INNER JOIN Service_Pkgs ON Orders.Pkg_ID=Service_Pkgs.ID
INNER JOIN Validity_Periods ON Orders.Validity_Period_ID=Validity_Periods.ID
GROUP BY Orders.Pkg_ID, Validity_Period_ID;
\end{lstlisting}

Similar to the one above, it requires an extra join to group by the selected validity period as well.
In the trigger it is still implemented by \verb OrderCompleted in a similar fashion as before, while the creation of the
entry in the \verb PurchasePerPackagePeriod table is delegated to the trigger \verb NewPeriod . \\

This reflects the fact that the Validity_Periods table is the one containing the join column with respect to the service
packages, and therefore validity periods are created after the service package they refer to: the \verb NewPackage trigger
would not have all the information needed to create an entry yet.

\subsection{Total value of sales per package with and without the optional products}

\begin{lstlisting}[style=SQL]
CREATE VIEW TotalPerPackage AS
SELECT
    Service_Pkgs.Name as Name,
    SUM(Validity_Periods.Monthly_Fee*Validity_Periods.Months) + SUM(Optional_Products.Monthly_Fee* Validity_Periods.Months) as Total,
    SUM(Validity_Periods.Monthly_Fee*Validity_Periods.Months) as "Total before optionals"
FROM Orders
    LEFT JOIN Service_Pkgs ON Orders.Pkg_ID=Service_Pkgs.ID
    LEFT JOIN Validity_Periods ON Orders.Validity_Period_ID=Validity_Periods.ID
    LEFT JOIN OrderComprehendsOptional ON Orders.ID=OrderComprehendsOptional.Order_ID
    LEFT JOIN Optional_Products ON OrderComprehendsOptional.Optional_ID=Optional_Products.ID
GROUP BY Orders.Pkg_ID;

\end{lstlisting}

All of the information included in an order have to be used to calculate the results of this view, making it one of the most complex.
The calculation is done from scratch, not using the calculated value in the \verb Total  column of the \verb Orders  table as to not
complicate the query further ( since it is not a value dependent on on the group by column, using it would have required a nested query).

The entry creation in the homonym table is handled by the trigger \verb NewPackage , while the updates is found in the \verb OrderCompleted trigger.
There, having the option of running two separate updates, the one for the complete total uses the value in the \verb Total  column for ease, while the
total before the optionals cost is computed by a simplified version of the query above.

\subsection{Average number of optional products sold together with each service package}

\begin{lstlisting}[style=SQL]
CREATE VIEW AvgOptPerPackage AS
SELECT Pkg_ID, Name, AVG(OptNum) FROM (
    SELECT Service_Pkgs.Name as Name, Service_Pkgs.ID as Pkg_ID, COUNT(coalesce(OrderComprehendsOptional.Optional_ID, 0)) as OptNum
        FROM Orders
            INNER JOIN Service_Pkgs ON Orders.Pkg_ID=Service_Pkgs.ID
            LEFT JOIN OrderComprehendsOptional ON Orders.ID=OrderComprehendsOptional.Order_ID
        GROUP BY Pkg_ID, Orders.ID) as Numbers
GROUP BY Pkg_ID;

\end{lstlisting}

The query uses a nested select statement to return an intermediate table where each row reflects an order and the number of optionals it includes.
The coalesce function is used to account for the NULL value that packages with no optionals would have in the join table. The average is then
computed on the values of this table.\\

The triggers relating to this table are the same ones as above. The query is reused pretty much untouched inside the \verb OrderCompleted
trigger to update to the new value the entry for each service package.

\subsection{List of insolvent users, suspended orders and alerts}

\begin{lstlisting}[style=SQL]
CREATE VIEW InsolventUsers AS
SELECT * FROM Users
WHERE Insolvent=True;

CREATE VIEW RejectedOrders AS
SELECT * FROM Orders
WHERE Status=’Failed’;
\end{lstlisting}

The views are pretty simple select statements. A view for the alerts has not been created because it would coincide exactly with the
whole \verb Audits  table.

All of the views are implemented as conditional statements in the \verb OrderFailed  trigger, while deletion of records from the tables are
inside the \verb OrderCompleted  trigger.

\subsection{Best seller optional product}

\begin{lstlisting}[style=SQL]
CREATE VIEW BestSellerOptional AS
SELECT
Optional_Products.Name as Name, COUNT(OrderComprehendsOptional.Order_ID) as TotalSales
FROM Optional_Products LEFT JOIN OrderComprehendsOptional ON OrderComprehendsOptional.Optional_ID=Optional_Products.ID
GROUP BY Optional_Products.ID
ORDER BY TotalSales DESC
LIMIT 1;
\end{lstlisting}

The view computes the number of sales per optional product and sorts it, returning only the highest record.\\

Having only one entry at any time, the updating of the materialized view is completely handled by the \verb OrderCompleted  trigger.

\section{Triggers}

This section illustrates the aforementioned triggers. All of the triggers rely on the \verb DELIMITER  command to change
the delimiting character in order to contain multiple statements.

\subsection {NewPackage and NewPeriod}


\begin{lstlisting}[style=SQL]
CREATE TRIGGER NewPackage
AFTER INSERT ON Service_Pkgs
FOR EACH ROW
BEGIN
    INSERT INTO PurchasesPerPackage VALUES ( new.ID,  new.Name, 0 );
    INSERT INTO TotalPerPackage VALUES ( new.ID,  new.Name, 0, 0 );
    INSERT INTO  AvgOptPerPackage VALUES ( new.ID,  new.Name, 0 );
END $$
\end{lstlisting}


\begin{lstlisting}[style=SQL]
CREATE TRIGGER NewPeriod
AFTER INSERT ON Validity_Periods
FOR EACH ROW
BEGIN
    INSERT INTO PurchasesPerPackagePeriod  (Pkg_ID, Name, Months, Purchases)
        SELECT Service_Pkgs.ID, Service_Pkgs.Name, new.Months AS Months, 0
        FROM Service_Pkgs WHERE ID = new.PKG_ID;
END $$
\end{lstlisting}

Both of these packages trigger after an insertion on the respective tables, and behave the same way, creating an entry
for the corresponding package or package-period combination inside the affected tables.

Consequently, these triggers fire only when an employee is configuring new packages, and only once per package. They
must act for each new record inserted, hence the for each row. Since if the trigger is firing the package is new, all
counts are set to zero.

\subsection{OrderCompleted}

\begin{lstlisting}[style=SQL]
CREATE TRIGGER OrderCompleted
AFTER UPDATE ON Orders
FOR EACH ROW
BEGIN

    IF new.Status="VALID" THEN
       DELETE FROM RejectedOrders WHERE ID=new.ID;
        INSERT INTO ServiceActivationSchedule VALUES(new.ID, new.User_ID, TIMESTAMPADD(MONTH,( SELECT Months FROM Validity_Periods WHERE ID=new.Validity_Period_ID ),new.Activation_Date));

        UPDATE PurchasesPerPackage set Purchases = Purchases + 1
        WHERE Pkg_ID=new.Pkg_ID;

        UPDATE PurchasesPerPackagePeriod set Purchases = Purchases + 1
        WHERE Pkg_ID=new.Pkg_ID AND Months =
            ( SELECT Months FROM Validity_Periods WHERE ID=new.Validity_Period_ID );

        UPDATE TotalPerPackage SET TotalPerPackage.Total = TotalPerPackage.Total + new.Total;

        UPDATE TotalPerPackage
            SET TotalPerPackage.TotalBeforeOptionals = TotalPerPackage.TotalBeforeOptionals +
            ( SELECT Validity_Periods.Monthly_Fee*Validity_Periods.Months
              FROM Orders
              LEFT JOIN Validity_Periods ON Orders.Validity_Period_ID=Validity_Periods.ID WHERE Orders.ID = new.ID
            );

        UPDATE AvgOptPerPackage SET AvgOptPerPackage.AvgOptionals =
            ( SELECT AVG(OptNum) FROM
                (SELECT Orders.ID as ID, Service_Pkgs.ID as Pkg_ID, COUNT(coalesce(OrderComprehendsOptional.Optional_ID, 0)) as OptNum
                    FROM Orders
                    INNER JOIN Service_Pkgs ON Orders.Pkg_ID=Service_Pkgs.ID
                    LEFT JOIN OrderComprehendsOptional ON Orders.ID=OrderComprehendsOptional.Order_ID
                    GROUP BY Pkg_ID, Orders.ID) as Numbers
                GROUP BY Pkg_ID
            );

        DELETE FROM BestSellerOptional;

        INSERT INTO BestSellerOptional (ID, Name, TotalSales) SELECT
            Optional_Products.ID as ID, Optional_Products.Name as Name, COUNT(OrderComprehendsOptional.Order_ID) as TotalSales
            FROM Optional_Products LEFT JOIN OrderComprehendsOptional ON OrderComprehendsOptional.Optional_ID=Optional_Products.ID
            GROUP BY Optional_Products.ID
            ORDER BY TotalSales DESC
            LIMIT 1;

        IF (SELECT COUNT(*) FROM Orders WHERE Status='FAILED' AND User_ID=new.User_ID ) = 0 THEN
            UPDATE Users SET Insolvent = FALSE, Failed_Payments = 0 WHERE ID=new.User_ID;
            DELETE FROM InsolventUsers WHERE ID=new.User_ID;
        END IF;
    END IF;
END $$

\end{lstlisting}



\subsection{OrderFailed}

\begin{lstlisting}[style=SQL]
CREATE TRIGGER OrderFailed
AFTER UPDATE ON Orders
FOR EACH ROW
BEGIN
    IF new.Status='FAILED' THEN

        IF new.ID NOT IN (SELECT ID FROM RejectedOrders) THEN
            INSERT INTO RejectedOrders VALUES (new.ID, new.User_ID, new.Pkg_ID, new.Timestamp, new.Status, new.Total);
        END IF;
        IF (SELECT Insolvent FROM Users WHERE ID=new.User_ID)=FALSE THEN
            UPDATE Users SET Insolvent=TRUE WHERE ID=new.User_ID;
            INSERT INTO InsolventUsers (ID, Mail, Username, Failed_Payments, Insolvent)
                SELECT ID, Mail, Username, Failed_Payments, Insolvent FROM Users WHERE ID=new.User_ID;
        END IF;

        UPDATE Users SET Failed_Payments = Failed_Payments + 1 WHERE ID=new.User_ID;
        UPDATE InsolventUsers SET Failed_Payments = Failed_Payments + 1 WHERE ID=new.User_ID;
        IF (SELECT Failed_Payments FROM Users WHERE ID=new.User_ID) = 3 THEN
            INSERT INTO Audits (User_ID, Mail, Username, Amount) SELECT ID, Mail, Username, new.Total FROM Users WHERE ID=new.User_ID;
        END IF;
    END IF;
END $$
\end{lstlisting}



\blindtext