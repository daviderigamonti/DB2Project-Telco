%---------------------------------------------------------------------
%   TRIGGERS CHAPTER
%---------------------------------------------------------------------

\chapter{Views and triggers}
\label{chap:views_triggers}

This section will address the behaviour of the triggers developed for the project, most of them were developed with the aim of operating on the materialized view tables (as per the project specification) but some include additional functionalities that we deliberately chose to implement via triggers such as the handling of the insolvent flag, failed payments, auditing mechanism and the service activation schedule creation.


\section{SQL View Code}
\label{sec:sql_views}

Since most of the triggers are related to the materialized views required for the sales report, this section prefaces the actual implementation with the code that would have been used for the actual views, had they not been materialized. Most of the code can still be reused in the actual triggers and the views share their name with their materialized counterpart for easier tracking.

\subsection{Number of total purchases per package}

\begin{lstlisting}[style=SQL]
CREATE VIEW PurchasesPerPackage AS
SELECT Name, COUNT(Pkg_ID) AS Purchases
FROM Orders INNER JOIN Service_Pkgs ON Orders.Pkg_ID=Service_Pkgs.ID
GROUP BY Pkg_ID;
\end{lstlisting}

In the triggers, it is mapped by \texttt{NewPackage} and \texttt{OrderCompleted}, which respectively create an entry in the \texttt{PurchasesPerPackage} table with zero purchases and increment it by one for each successful order containing the package.

\subsection{Number of total purchases per package and validity period}

\begin{lstlisting}[style=SQL]
CREATE VIEW PurchasePerPackagePeriod AS
SELECT  Name,
        Validity_Periods.Months AS Months,
        COUNT(Orders.Pkg_ID) AS Purchases
FROM Orders
INNER JOIN Service_Pkgs ON Orders.Pkg_ID=Service_Pkgs.ID
INNER JOIN Validity_Periods ON Orders.Validity_Period_ID=Validity_Periods.ID
GROUP BY Orders.Pkg_ID, Validity_Period_ID;
\end{lstlisting}

Similar to the one above, it requires an extra join to group by the selected validity period as well. In the trigger it is still implemented by \texttt{OrderCompleted} in a similar fashion as before, while the creation of the entry in the \texttt{PurchasePerPackagePeriod} table is delegated to the trigger \texttt{NewPeriod}.

This reflects the fact that the Validity\_Periods table is the one containing the join column with respect to the service packages, and therefore validity periods are created after the service package they refer to: the \texttt{NewPackage} trigger would not have all the information needed to create an entry yet.

\subsection{Total value of sales per package with and without the optional products}

\begin{lstlisting}[style=SQL]
CREATE VIEW TotalPerPackage AS
SELECT
    Service_Pkgs.Name as Name,
    SUM(Validity_Periods.Monthly_Fee*Validity_Periods.Months) +
        SUM(Optional_Products.Monthly_Fee* Validity_Periods.Months) as Total,
    SUM(Validity_Periods.Monthly_Fee*Validity_Periods.Months)
        as "Total before optionals"
FROM Orders
    LEFT JOIN Service_Pkgs ON Orders.Pkg_ID=Service_Pkgs.ID
    LEFT JOIN Validity_Periods
        ON Orders.Validity_Period_ID=Validity_Periods.ID
    LEFT JOIN OrderComprehendsOptional
        ON Orders.ID=OrderComprehendsOptional.Order_ID
    LEFT JOIN Optional_Products
        ON OrderComprehendsOptional.Optional_ID=Optional_Products.ID
GROUP BY Orders.Pkg_ID;

\end{lstlisting}

All of the information included in an order have to be used to calculate the results of this view, making it one of the most complex. The calculation is done from scratch, not using the calculated value in the \texttt{Total}  column of the \texttt{Orders}  table as to not complicate the query further (since it is not a value that depends on the GROUP BY column, using it would have required a nested query).

The entry creation in the homonym table is handled by the trigger \texttt{NewPackage}, while the updates is found in the \texttt{OrderCompleted} trigger. There, having the option of running two separate updates, the one for the complete total uses the value in the \texttt{Total} column for ease, while the total before the optionals cost is computed by a simplified version of the query above.

\subsection{Average number of optional products sold together with each service package}

\begin{lstlisting}[style=SQL]
CREATE VIEW AvgOptPerPackage AS
SELECT Pkg_ID, Name, AVG(OptNum) FROM (
    SELECT Service_Pkgs.Name as Name, Service_Pkgs.ID as Pkg_ID,
    COUNT(coalesce(OrderComprehendsOptional.Optional_ID, 0)) as OptNum
        FROM Orders
            INNER JOIN Service_Pkgs ON Orders.Pkg_ID=Service_Pkgs.ID
            LEFT JOIN OrderComprehendsOptional
                ON Orders.ID=OrderComprehendsOptional.Order_ID
        GROUP BY Pkg_ID, Orders.ID) as Numbers
GROUP BY Pkg_ID;
\end{lstlisting}

The query uses a nested select statement to return an intermediate table where each row reflects an order and the number of optionals it includes.
The coalesce function is used to account for the NULL value that packages with no optionals would have in the join table. The average is then computed on the values of this table.

The triggers relating to this table are the same ones as above. The query is reused pretty much untouched inside the \texttt{OrderCompleted}
trigger to update to the new value the entry for each service package.

\subsection{List of insolvent users, suspended orders and alerts}

\begin{lstlisting}[style=SQL]
CREATE VIEW InsolventUsers AS
SELECT * FROM Users
WHERE Insolvent=True;

CREATE VIEW RejectedOrders AS
SELECT * FROM Orders
WHERE Status='Failed';
\end{lstlisting}

The views are pretty simple select statements. A view for the alerts has not been created because it would coincide exactly with the whole \texttt{Audits} table.

All of the views are implemented as conditional statements in the \texttt{OrderFailed}  trigger, while deletion of records from the tables are inside the \texttt{OrderCompleted} trigger.

\subsection{Best seller optional product}

\begin{lstlisting}[style=SQL]
CREATE VIEW BestSellerOptional AS
SELECT
    Optional_Products.Name as Name,
    COUNT(OrderComprehendsOptional.Order_ID) as TotalSales
FROM Optional_Products
    LEFT JOIN OrderComprehendsOptional
        ON OrderComprehendsOptional.Optional_ID=Optional_Products.ID
GROUP BY Optional_Products.ID
ORDER BY TotalSales DESC
LIMIT 1;
\end{lstlisting}

The view computes the number of sales per optional product and sorts it, returning only the highest record.

Having only one entry at any time, the updating of the materialized view is completely handled by the \texttt{OrderCompleted} trigger.


\section{Triggers}
\label{sec:triggers}

This section illustrates the aforementioned triggers. All of the triggers rely on the \texttt{DELIMITER \$\$} command to change the delimiting character in order to contain multiple statements.

Also, all the triggers here are row level triggers - while this would have been the choice anyway, since each trigger acts on the data of a single record, MySQL does not support statement level triggers, so there was no other alternative.

\subsection {NewPackage and NewPeriod}


\begin{lstlisting}[style=SQL]
CREATE TRIGGER NewPackage
AFTER INSERT ON Service_Pkgs
FOR EACH ROW
BEGIN
    INSERT INTO PurchasesPerPackage VALUES ( new.ID,  new.Name, 0 );
    INSERT INTO TotalPerPackage VALUES ( new.ID,  new.Name, 0, 0 );
    INSERT INTO AvgOptPerPackage VALUES ( new.ID,  new.Name, 0 );
END $$
\end{lstlisting}

The last trigger fires on the same condition as the one above: after an update of orders. They have been separated both for readability, as they are the longest triggers in the project already, and for logical reasons: while \texttt{OrderCompleted} fires upon an order being marked valid, this one deals with what should happen when a payment is rejected.

\begin{lstlisting}[style=SQL]
CREATE TRIGGER NewPeriod
AFTER INSERT ON Validity_Periods
FOR EACH ROW
BEGIN
    INSERT INTO PurchasesPerPackagePeriod  (Pkg_ID, Name, Months, Purchases)
        SELECT Service_Pkgs.ID, Service_Pkgs.Name, new.Months AS Months, 0
        FROM Service_Pkgs WHERE ID = new.PKG_ID;
END $$
\end{lstlisting}

Both of these packages trigger after an insertion on the respective tables, and behave the same way, creating an entry for the corresponding package or package-period combination inside the affected tables.

Consequently, these triggers fire only when an employee is configuring new packages, and only once per package. They must act for each new record inserted, hence the for each row. Since if the trigger is firing the package is new, all counts are set to zero.

The trigger \texttt{NewPeriod} also implicitly enforces the constraint that no package has two validity periods of the same length, as trying to insert one such period would raise a duplicate primary key error on the \texttt{PurchasesPerPacakgePeriod} table.

\subsection{OrderCompleted}

\begin{lstlisting}[style=SQL]
CREATE TRIGGER OrderCompleted
AFTER UPDATE ON Orders
FOR EACH ROW
BEGIN
    IF new.Status='VALID' AND old.STATUS<>'VALID' THEN
        DELETE FROM SuspendedOrders WHERE ID=new.ID;
        INSERT INTO ServiceActivationSchedule
            VALUES(new.ID, new.User_ID,
            TIMESTAMPADD(MONTH, (SELECT Months FROM Validity_Periods
                WHERE ID=new.Validity_Period_ID ),
            new.Activation_Date)
        );

        UPDATE PurchasesPerPackage set Purchases = Purchases + 1
        WHERE Pkg_ID=new.Pkg_ID;

        UPDATE PurchasesPerPackagePeriod set Purchases = Purchases + 1
        WHERE Pkg_ID=new.Pkg_ID AND Months =
            ( SELECT Months FROM Validity_Periods WHERE ID=new.Validity_Period_ID );

        UPDATE TotalPerPackage SET TotalPerPackage.Total =
            TotalPerPackage.Total + new.Total WHERE Pkg_ID=new.Pkg_ID;

        UPDATE TotalPerPackage
            SET TotalPerPackage.TotalBeforeOptionals =
                TotalPerPackage.TotalBeforeOptionals +
                ( SELECT Validity_Periods.Monthly_Fee*Validity_Periods.Months
                    FROM Orders
                    LEFT JOIN Validity_Periods
                        ON Orders.Validity_Period_ID=Validity_Periods.ID
                        WHERE Orders.ID = new.ID
                ) WHERE Pkg_ID=new.Pkg_ID;

        UPDATE AvgOptPerPackage SET AvgOptPerPackage.AvgOptionals =
            ( SELECT AVG(OptNum) FROM
                (SELECT Orders.ID as ID, Service_Pkgs.ID as Pkg_ID,
                    COUNT(coalesce(OrderComprehendsOptional.Optional_ID, 0)) 
                        as OptNum
                    FROM Orders
                    INNER JOIN Service_Pkgs ON Orders.Pkg_ID=Service_Pkgs.ID
                    LEFT JOIN OrderComprehendsOptional
                        ON Orders.ID=OrderComprehendsOptional.Order_ID
                    GROUP BY Pkg_ID, Orders.ID) as Numbers
                GROUP BY Pkg_ID HAVING Pkg_ID=new.Pkg_ID
            ) WHERE Pkg_ID=new.Pkg_ID;


        DELETE FROM BestSellerOptional;

        INSERT INTO BestSellerOptional (ID, Name, TotalSales) SELECT
            Optional_Products.ID as ID, Optional_Products.Name as Name,
            COUNT(OrderComprehendsOptional.Order_ID) as TotalSales
            FROM Optional_Products
            LEFT JOIN OrderComprehendsOptional
                ON OrderComprehendsOptional.Optional_ID=Optional_Products.ID
            GROUP BY Optional_Products.ID
            ORDER BY TotalSales DESC LIMIT 1;


        IF (
            SELECT COUNT(*) FROM Orders WHERE
                Status='FAILED' AND User_ID=new.User_ID ) = 0
        THEN
            UPDATE Users SET Insolvent = FALSE, Failed_Payments = 0
                WHERE ID=new.User_ID;
            DELETE FROM InsolventUsers WHERE ID=new.User_ID;
        END IF;
    END IF;
END $$

\end{lstlisting}

This is the biggest trigger in the project, as it handles everything that happens when an order is marked valid. Only orders that have been paid for are taken into account for the statistics shown in the sales report, therefore this trigger handles the insertion of data in most of the materialized views, and the removal of data from the list of rejected orders and insolvent users.

Since \texttt{BestSellerOptional} always contains only one row, the trigger deletes the whole table before computing the new best seller.

Furthermore, the trigger handles the \texttt{ServiceActivationSchedule} table, as this was the most natural way to implement the insertion of data into that table.

\subsection{OrderFailed}

\begin{lstlisting}[style=SQL]
CREATE TRIGGER OrderFailed
AFTER UPDATE ON Orders
FOR EACH ROW
BEGIN
    IF new.Status='FAILED' THEN
        IF new.ID NOT IN (SELECT ID FROM SuspendedOrders) THEN
            INSERT INTO SuspendedOrders VALUES (
            new.ID, new.User_ID, new.Pkg_ID, new.Timestamp, new.Status, new.Total);
        ELSE
            UPDATE SuspendedOrders SET Status='FAILED' WHERE ID=new.ID;
        END IF;

        IF (SELECT Insolvent FROM Users WHERE ID=new.User_ID)=FALSE THEN
            UPDATE Users SET Insolvent=TRUE WHERE ID=new.User_ID;
            INSERT INTO InsolventUsers (
            ID, Mail, Username, Failed_Payments, Insolvent)
                SELECT ID, Mail, Username, Failed_Payments, Insolvent
                FROM Users WHERE ID=new.User_ID;
        END IF;

        UPDATE Users SET Failed_Payments = Failed_Payments + 1 WHERE ID=new.User_ID;
        UPDATE InsolventUsers SET Failed_Payments = Failed_Payments + 1
            WHERE ID=new.User_ID;
        IF (SELECT Failed_Payments FROM Users WHERE ID=new.User_ID) = 3 THEN
            INSERT INTO Audits (User_ID, Mail, Username, Amount)
            SELECT ID, Mail, Username, new.Total FROM Users
            WHERE ID=new.User_ID;
        END IF;
    END IF;
END $$
\end{lstlisting}

Other than handling the aforementioned \texttt{InsolventUsers} and \texttt{SuspendedOrders}, the auditing mechanism is also implemented here: upon a user failing their third payment, an entry is created in the \texttt{Audits} table.

