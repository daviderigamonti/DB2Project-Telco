%---------------------------------------------------------------------
%   ORM SECTION
%---------------------------------------------------------------------

\chapter{Object Relational Mapping (ORM)}
\label{chap:orm}


\section{ORM relationship design}
\label{sec:rel_design}

\subsection*{Relationship 'Owns'}

\begin{minipage}[h]{0.5\textwidth}
    \begin{figure}[H]
        \includesvg[scale=0.69]{img/ORM_User_SAC.svg}
        \caption{'Owns' relationship}
        \label{fig:orm_u_sac}
    \end{figure}
\end{minipage}
\hfill
\begin{minipage}[h]{0.48\textwidth}
    \begin{itemize}
        \item \textbf{User → ServiceActivationSchedule} \texttt{@OneToMany} is necessary to list the activation schedules for a given user; fetch mode is \texttt{LAZY} since the activation schedule will not always be consulted when loading a user (e.g. during login procedure).
        \item \textbf{ServiceActivationSchedule → User} \texttt{@ManytoOne} is not required by the specification, however, mapping the relationship as bidirectional and using only the needed side is preferable; fetch mode is therefore \texttt{LAZY}.
    \end{itemize}
\end{minipage}

\subsection*{Relationship 'Places'}

\begin{minipage}[h]{0.5\textwidth}
    \begin{figure}[H]
        \includesvg[scale=0.69]{img/ORM_User_Order.svg}
        \caption{'Places' relationship}
        \label{fig:orm_u_o}
    \end{figure}
\end{minipage}
\hfill
\begin{minipage}[h]{0.48\textwidth}
    \begin{itemize}
        \item \textbf{User → Order} \texttt{@OneToMany} is necessary to list the orders that have been made by a user but fetch mode is \texttt{LAZY}, since it is unnecessary to load all the orders of a given user every time this user is fetched from the database.
        \item \textbf{Order → User} \texttt{@ManyToOne} is necessary since the order has to contain the information about its user; fetch mode is \texttt{LAZY} because the information about the user is not always relevant when viewing an order (e.g. listing the rejected orders for a user).
    \end{itemize}
\end{minipage}

\subsection*{Relationship 'Schedules'}

\begin{minipage}[h]{0.5\textwidth}
    \begin{figure}[H]
        \includesvg[scale=0.69]{img/ORM_SAC_Order.svg}
        \caption{'Schedules' relationship}
        \label{fig:orm_sac_o}
    \end{figure}
\end{minipage}
\hfill
\begin{minipage}[h]{0.48\textwidth}
    \begin{itemize}
        \item \textbf{ServiceActivationSchedule → Order} \texttt{@OneToOne} is necessary because the schedule includes the service package and the optional packages from the order and fetch mode is \texttt{EAGER} because without the order, the schedule information is incomplete.
        \item The relationship is unidirectional, since the order doesn’t ever need to retrieve the activation schedule.
    \end{itemize}
\end{minipage}

\subsection*{Relationship 'Comprises'}

\begin{minipage}[h]{0.5\textwidth}
    \begin{figure}[H]
        \includesvg[scale=0.69]{img/ORM_Order_SP.svg}
        \caption{'Comprises' relationship}
        \label{fig:orm_o_sp}
    \end{figure}
\end{minipage}
\hfill
\begin{minipage}[h]{0.48\textwidth}
    \begin{itemize}
        \item \textbf{Order → ServicePackage} \texttt{@OneToOne} is necessary because the service package contains all the information about the services related to the order; fetch mode is \texttt{LAZY} for conservative reasons, however it might also have been \texttt{EAGER} but when in doubt it is good practice to pick \texttt{LAZY} in order to not load unnecessary information every time an order is retrieved.
        \item The relationship is unidirectional since the only data required by the dashboard is aggregated and can be derived via a simple GROUP BY query.
    \end{itemize}
\end{minipage}

\subsection*{Relationship 'Comprehends'}

\begin{minipage}[h]{0.5\textwidth}
    \begin{figure}[H]
        \includesvg[scale=0.69]{img/ORM_Order_OP.svg}
        \caption{'Comprehends' relationship}
        \label{fig:orm_o_op}
    \end{figure}
\end{minipage}
\hfill
\begin{minipage}[h]{0.48\textwidth}
    \begin{itemize}
        \item \textbf{Order → OptionalProduct} \\ \texttt{@OneToMany} is necessary for each order to contain information about the selected optional products; fetch mode is \texttt{LAZY} for the same reasons listed in the previous relationship.
        \item The relationship is unidirectional since the optional product does not need to know which orders it is part of.
        \item Since the mapped relationship is \texttt{Many to many} the generated join table is called \texttt{OrderComprehendsOptional}
    \end{itemize}
\end{minipage}

\subsection*{Relationship 'Lasts'}

\begin{minipage}[h]{0.5\textwidth}
    \begin{figure}[H]
        \includesvg[scale=0.69]{img/ORM_Order_VP.svg}
        \caption{'Lasts' relationship}
        \label{fig:orm_o_vp}
    \end{figure}
\end{minipage}
\hfill
\begin{minipage}[h]{0.48\textwidth}
    \begin{itemize}
        \item \textbf{Order → ValidityPeriod} \texttt{@OneToOne} is necessary for each order to be associated with one validity period; fetch mode is \texttt{LAZY} because the information is used only inside the summary.
        \item The relationship is unidirectional since the validity period does not need to know in which orders it is used.
    \end{itemize}
\end{minipage}

\subsection*{Relationship 'Offers'}

\begin{minipage}[h]{0.5\textwidth}
    \begin{figure}[H]
        \includesvg[scale=0.69]{img/ORM_SP_OP.svg}
        \caption{'Offers' relationship}
        \label{fig:orm_sp_op}
    \end{figure}
\end{minipage}
\hfill
\begin{minipage}[h]{0.48\textwidth}
    \begin{itemize}
        \item \textbf{ServicePackage → OptionalProduct} \texttt{@OneToMany} is necessary for each service package to be associated with the optional products it includes; fetch mode is \texttt{LAZY}, because the information about which optional products are available is relevant only during the subscription and creation phase.
        \item The relationship is unidirectional since there is no need to actively track in which service packages is offered a given optional product.
    \end{itemize}
\end{minipage}

\subsection*{Relationship 'Provides'}

\begin{minipage}[h]{0.5\textwidth}
    \begin{figure}[H]
        \includesvg[scale=0.69]{img/ORM_SP_VP.svg}
        \caption{'Provides' relationship}
        \label{fig:orm_sp_vp}
    \end{figure}
\end{minipage}
\hfill
\begin{minipage}[h]{0.48\textwidth}
    \begin{itemize}
        \item \textbf{ServicePackage → ValidityPeriod} \\ \texttt{@OneToMany} is necessary for each service package to be associated with the possible validity periods that the user may pick; fetch mode is \texttt{LAZY} since the information about which validity periods are available is relevant only during the subscription and creation phase, in addition, the cascading type is set to \texttt{PERSIST} in order to automatically create the validity period when we persist the service package.
        \item The relationship is unidirectional since there is no need to actively track in which service packages is offered a given validity period.
    \end{itemize}
\end{minipage}

\subsection*{Relationship 'Includes'}

\begin{minipage}[h]{0.5\textwidth}
    \begin{figure}[H]
        \includesvg[scale=0.69]{img/ORM_SP_Service.svg}
        \caption{'Includes' relationship}
        \label{fig:orm_sp_s}
    \end{figure}
\end{minipage}
\hfill
\begin{minipage}[h]{0.48\textwidth}
    In order to avoid replicating three times the same relationship, the generic \texttt{'Service'} represents the three types of services: \texttt{MobilePhone}, \texttt{FixedPhone} and \texttt{Internet}.
    \begin{itemize}
        \item \textbf{ServicePackage → Service} \\ \texttt{@OneToMany} is necessary for each service package to be associated with the services it contains; fetch mode is \texttt{EAGER} because this is the main information inside the service package and it would be pointless to load it without its services, in addition, the cascading type is set to \texttt{PERSIST} in order to automatically create the services when we persist the service package. 
        \item \textbf{Service → ServicePackage} \\ \texttt{@ManytoOne} is not required by the specification, however, mapping the relationship as bidirectional and using only the needed side is preferable; fetch mode is therefore \texttt{LAZY} by default.
    \end{itemize}
\end{minipage}


\section{Java entities}
\label{sec:java_ent}

Only the entities that represent proper independent tables inside the database will be reported in this section, since entities derived from materialized views don't have any particular or interesting JPA annotations and have already been extensively covered in the \hyperref[chap:views_triggers]{Triggers chapter}.

\subsection*{Audit}

\begin{lstlisting}[style = JPA]
\@@Entity@@/
\@@IdClass@@/(AuditID.class)
\@@Table@@/(name= "Audits", schema = "db2telco")
\@@NamedQuery@@/(name = "Audit.findAll", query = "SELECT a FROM Audit a")
public class Audit implements Serializable {
    private static final long serialVersionUID = 1L;

    \@@Id@@/
    \@@Column@@/(name = "User_ID")
    private int userID;

    \@@Id@@/
    \@@Column@@/(name = "Timestamp")
    private Timestamp timestamp;

    \@@Column@@/(name="Mail")
    private String mail;

    \@@Column@@/(name="Username")
    private String username;

    \@@Column@@/(name = "Amount")
    private float amount;

    ...

}
\end{lstlisting}

\subsection*{Employee}

\begin{lstlisting}[style = JPA]
\@@Entity@@/
\@@Table@@/(name= "Employees", schema = "db2telco")
\@@NamedQuery@@/(name = "Employee.findByUsername", query = "SELECT u " + 
                                                      "FROM Employee u " + 
                                                      "WHERE u.username = :usr"
)
public class Employee implements Serializable {
    private static final long serialVersionUID = 1L;

    \@@Id@@/
    \@@GeneratedValue@@/(strategy = GenerationType.IDENTITY)
    \@@Column@@/(name = "ID")
    private int id;

    \@@Column@@/(name="Username")
    private String username;

    \@@Column@@/(name="Password")
    \@@JsonIgnore@@/
    private String password;

    ...

}
\end{lstlisting}

\subsection*{Fixed phone}

\begin{lstlisting}[style = JPA]
\@@Entity@@/
\@@Table@@/(name= "Fixed_Phone", schema = "db2telco")
public class FixedPhone implements Serializable {
    private static final long serialVersionUID = 1L;

    \@@Id@@/
    \@@GeneratedValue@@/(strategy = GenerationType.IDENTITY)
    \@@Column@@/(name = "ID")
    private int id;

    \@@ManyToOne@@/
    \@@JoinColumn@@/(name = "Pkg_ID")
    \@@JsonBackReference@@/
    private ServicePackage servicePackage;

    ...

}
\end{lstlisting}

\subsection*{Internet}

\begin{lstlisting}[style = JPA]
\@@Entity@@/
\@@Table@@/(name= "Internet", schema = "db2telco")
public class Internet implements Serializable {
    private static final long serialVersionUID = 1L;

    \@@Id@@/
    \@@GeneratedValue@@/(strategy = GenerationType.IDENTITY)
    \@@Column@@/(name = "ID")
    private int id;

    \@@ManyToOne@@/
    \@@JoinColumn@@/(name = "Pkg_ID")
    \@@JsonBackReference@@/
    private ServicePackage servicePackage;

    \@@Column@@/(name="GB_N")
    private int gigabytes;

    \@@Column@@/(name="GB_Fee")
    private float gigabyteFee;

    \@@Column@@/(name="Fixed")
    private boolean is_fixed;
    
    ...

}
\end{lstlisting}

\subsection*{Mobile phone}

\begin{lstlisting}[style = JPA]
\@@Entity@@/
\@@Table@@/(name= "Mobile_Phone", schema = "db2telco")
public class MobilePhone implements Serializable {
    private static final long serialVersionUID = 1L;

    \@@Id@@/
    \@@GeneratedValue@@/(strategy = GenerationType.IDENTITY)
    \@@Column@@/(name = "ID")
    private int id;

    \@@ManyToOne@@/
    \@@JoinColumn@@/(name = "Pkg_ID")
    \@@JsonBackReference@@/
    private ServicePackage servicePackage;

    \@@Column@@/(name="Minutes_N")
    private int minutes;

    \@@Column@@/(name="SMS_N")
    private int sms;

    \@@Column@@/(name="Minutes_Fee")
    private float minuteFee;

    \@@Column@@/(name="SMS_Fee")
    private float smsFee;
    
    ...

}
\end{lstlisting}

\subsection*{Optional product}

\begin{lstlisting}[style = JPA]
\@@Entity@@/
\@@Table@@/(name= "Optional_Products", schema = "db2telco")
\@@NamedQuery@@/(name = "OptionalProduct.findAll", query = "SELECT s " + 
                                                      "FROM OptionalProduct s"
)
public class OptionalProduct implements Serializable {
    private static final long serialVersionUID = 1L;

    \@@Id@@/
    \@@GeneratedValue@@/(strategy = GenerationType.IDENTITY)
    \@@Column@@/(name = "ID")
    private int id;

    \@@Column@@/(name="Name")
    private String \!!name!!/;

    \@@Column@@/(name="Monthly_Fee")
    private float fee;
    
    ...

}
\end{lstlisting}

\subsection*{Order}

\begin{lstlisting}[style = JPA]
\@@Entity@@/
\@@Table@@/(name= "Orders", schema = "db2telco")
\@@NamedQuery@@/(name = "Order.getRejectedOrdersByUser", 
            query = "SELECT o FROM Order o " +
                    "WHERE o.user.id = :id AND " +
                    "(o.status = " +
    "it.polimi.db2project_telco.server.entities.util.OrderStatus.FAILED OR " +
                    "o.status = " +
    "it.polimi.db2project_telco.server.entities.util.OrderStatus.PENDING)"
)
public class Order implements Serializable {
    private static final long serialVersionUID = 1L;

    \@@Id@@/
    \@@GeneratedValue@@/(strategy = GenerationType.IDENTITY)
    \@@Column@@/(name = "ID")
    private int id;

    \@@ManyToOne@@/(fetch = FetchType.LAZY)
    \@@JoinColumn@@/(name = "User_ID")
    private User user;

    \@@OneToOne@@/(fetch = FetchType.LAZY)
    \@@JoinColumn@@/(name = "Pkg_ID")
    private ServicePackage servicePackage;

    \@@OneToOne@@/(fetch = FetchType.LAZY)
    \@@JoinColumn@@/name = "Validity_Period_ID")
    private ValidityPeriod validityPeriod;

    \@@OneToMany@@/(fetch = FetchType.LAZY)
    \@@JoinTable@@/(name="OrderComprehendsOptional",
            joinColumns = \@@JoinColumn@@/(name = "Order_ID"),
            inverseJoinColumns = \@@JoinColumn@@/(name = "Optional_ID")
    )
    private List<OptionalProduct> optionalProducts;

    \@@Column@@/(name="Activation_Date")
    private Date activationDate;

    \@@Column@@/(name="Timestamp")
    private Timestamp timestamp;

    \@@Column@@/(name="Status")
    \@@Enumerated@@/(EnumType.STRING)
    private OrderStatus status;

    \@@Column@@/(name="Total")
    private float total;
    
    ...

}
\end{lstlisting}

\subsection*{Service activation schedule}

\begin{lstlisting}[style = JPA]
\@@Entity@@/
\@@IdClass@@/(ScheduleID.class)
\@@Table@@/(name= "ServiceActivationSchedule", schema = "db2telco")
public class ServiceActivationSchedule implements Serializable {
    private static final long serialVersionUID = 1L;

    \@@Id@@/
    \@@ManyToOne@@/(fetch = FetchType.LAZY)
    \@@JoinColumn@@/(name = "User_ID")
    private User user;

    \@@Id@@/
    \@@OneToOne@@/(fetch = FetchType.EAGER)
    \@@JoinColumn@@/(name = "Order_ID")
    private Order order;

    \@@Column@@/(name = "Deactivation_Date")
    private Date deactivationDate;
    
    ...

}
\end{lstlisting}

\subsection*{Service package}

\begin{lstlisting}[style = JPA]
\@@Entity@@/
\@@Table@@/(name= "Service_Pkgs", schema = "db2telco")
\@@NamedQuery@@/(name = "ServicePackage.findAll", query = "SELECT s " +
                                                     "FROM ServicePackage s"
)
public class ServicePackage implements Serializable {
    private static final long serialVersionUID = 1L;

    \@@Id@@/
    \@@GeneratedValue@@/(strategy = GenerationType.IDENTITY)
    \@@Column@@/(name = "ID")
    private int id;

    \@@Column@@/(name="Name")
    private String name;

    \@@OneToMany@@/(mappedBy = "servicePackage", fetch = FetchType.EAGER, 
               cascade = CascadeType.PERSIST)
    \@@JsonManagedReference@@/
    private List<FixedPhone> fixedPhoneServices;

    \@@OneToMany@@/(mappedBy = "servicePackage", fetch = FetchType.EAGER, 
               cascade = CascadeType.PERSIST)
    \@@JsonManagedReference@@/
    private List<MobilePhone> mobilePhoneServices;

    \@@OneToMany@@/(mappedBy = "servicePackage", fetch = FetchType.EAGER,
               cascade = CascadeType.PERSIST)
    \@@JsonManagedReference@@/
    private List<Internet> internetServices;

    \@@OneToMany@@/(mappedBy = "servicePackage", fetch = FetchType.LAZY,
               cascade = CascadeType.PERSIST)
    \@@JsonManagedReference@@/
    private List<ValidityPeriod> validityPeriods;

    \@@OneToMany@@/(fetch = FetchType.LAZY)
    \@@JoinTable@@/(name="ServicePkgOffersOptional",
            joinColumns = \@@JoinColumn@@/(name = "Pkg_ID"),
            inverseJoinColumns = \@@JoinColumn@@/(name = "Optional_ID")
    )
    private List<OptionalProduct> optionalProducts;
    
    ...

}
\end{lstlisting}

\subsection*{User}

\begin{lstlisting}[style = JPA]
\@@Entity@@/
\@@Table@@/(name= "Users", schema = "db2telco")
\@@NamedQuery@@/(name = "User.findByUsername", query = "SELECT u " + 
                                                  "FROM User u " + 
                                                  "WHERE u.username = :usr"
)
public class User implements Serializable {
    private static final long serialVersionUID = 1L;

    \@@Id@@/
    \@@GeneratedValue@@/(strategy = GenerationType.IDENTITY)
    \@@Column@@/(name = "ID")
    private int id;

    \@@Column@@/(name="Mail")
    private String mail;

    \@@Column@@/(name="Username")
    private String username;

    \@@Column@@/(name="Password")
    \@@JsonIgnore@@/
    private String password;

    \@@Column@@/(name="Failed_Payments")
    \@@JsonIgnore@@/
    private int failed_payments;

    \@@Column@@/(name="Insolvent")
    \@@JsonIgnore@@/
    private boolean is_insolvent;

    \@@OneToMany@@/(mappedBy = "user", fetch = FetchType.LAZY)
    private List<Order> orders;

    \@@OneToMany@@/(mappedBy = "user", fetch = FetchType.LAZY)
    private List<ServiceActivationSchedule> schedules;
    
    ...

}
\end{lstlisting}

\subsection*{Validity period}

\begin{lstlisting}[style = JPA]
\@@Entity@@/
\@@Table@@/(name= "Validity_Periods", schema = "db2telco")
public class ValidityPeriod implements Serializable {
    private static final long serialVersionUID = 1L;

    \@@Id@@/
    \@@GeneratedValue@@/(strategy = GenerationType.IDENTITY)
    \@@Column@@/(name = "ID")
    private int id;

    \@@ManyToOne@@/
    \@@JoinColumn@@/(name = "Pkg_ID")
    \@@JsonBackReference@@/
    private ServicePackage servicePackage;

    \@@Column@@/(name="Months")
    private int months;

    \@@Column@@/(name="Monthly_Fee")
    private float fee;
    
    ...

}
\end{lstlisting}
