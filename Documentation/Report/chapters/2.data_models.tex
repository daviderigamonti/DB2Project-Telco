%---------------------------------------------------------------------
%   DATA MODELS CHAPTER
%---------------------------------------------------------------------

\chapter{Conceptual and logical data models}
\label{chap:data_models}


\section{ER Diagram}
\label{sec:er_diagram}

\begin{figure}[h]
    \centering
    \centerline{\includesvg[scale=0.45]{img/ER_diagram.svg}}
    \caption{ER diagram for the database}
    \label{fig:er_diagram}
\end{figure}

The proposed ER diagram reflects the directives indicated in the given specifications.

In some entities we deemed appropriate to add a generated incremental ID as the primary key even if there were other candidate keys available (e.g. the username of the \textit{User} could have been its primary key), this was done in order to simplify foreign key handling with the added benefit of being more maintainable in the long term.

Another doubt that arose during the planning phase of the database was related to the \textit{Fixed Phone} entity since, in the specifications, it was clearly stated that it shouldn't have any particular configuration parameters, therefore modeling it as an entity would mean having an "empty" entity; in the end we decided to stick with this choice for the sake of consistency with the other services and because it is also a scalable solution in case the company needs to add parameters to the service in the future.

The specifications document is not completely transparent regarding the \textit{Service Activation Schedule} and its purpose inside the database: although it is stated that the schedule must include the date of activation and deactivation for the services and the optional products for a given user, we opted to include just a reference to the user, a reference to the order and the deactivation date.

The previous choice was taken in order to reduce redundancy and minimize the overhead in terms of computation time and disk space for each \textit{Order} (in a real case scenario, ideally, the company would have a high number of orders) without compromising the available information, since we can retrieve all of the specified data from the order reference due to the fact that the services and optional products share the order's validity period.

The specifications document esplicitly defines the \textit{Validity Periods} available for a certain \textit{Service Package} as 12 months, 24 months and 36 months, however, we found this model quite reductive and decided to generalize the number of possible \textit{Validity Periods} from 1 to N.

It's important to notice that, when an \textit{Employee} tries to create a new \textit{Service Package} from the designated interface, the default \textit{Validity Periods} will be the original three options specified in the document and they will have the ability to add or remove \textit{Validity Periods} to their liking.


\section{Logical model}
\label{sec:logical_model}

What follows here is the definition of the tables in the database. It adheres to the ER diagram proposed above, resolving the ISA of the various service types into three different tables (\texttt{Internet, Mobile\_Phone, Fixed\_Phone}). A hybrid approach has been chosen, simplifying the two kinds of internet services into one, but leaving the other services as separate. This was a compromise between keeping the tables slim and the database scalable and optimizing the complexity away into a single table.

Simple constraints and checks are already enforced at this level, while some more complex one are left to the triggers or to java (as checking that the optionals and validity period selected for a package are indeed in the list of features that package offers).

The code relating to the materialized views is omitted, as they are simple mirrors of the values present in this table and contain no checks or foreign key constraints, and their workings will be described in the trigger section.

As for the foreign key constraint, we have imagined that when a service package is deleted, all the associated services would be deleted as well, but the orders related to it would remain, to keep a trace of the financial movements. Conversely, if a user were to be deleted, we assume that all of their data, including past transaction, is no longer relevant, and delete their orders as well.

\begin{lstlisting}[style=SQL]
CREATE TABLE IF NOT EXISTS Users (
    ID INT NOT NULL AUTO_INCREMENT,
    Mail varchar(255) UNIQUE NOT NULL,
    Username varchar(255) UNIQUE NOT NULL,
    Password varchar(64) NOT NULL,
    Failed_Payments INT DEFAULT 0 NOT NULL CHECK (Failed_Payments  >= 0),
    Insolvent BOOLEAN DEFAULT FALSE NOT NULL,
    PRIMARY KEY ( ID )
);

CREATE TABLE IF NOT EXISTS Service_Pkgs (
    ID INT NOT NULL AUTO_INCREMENT,
    Name varchar(255) NOT NULL,
    PRIMARY KEY ( ID )
);

CREATE TABLE IF NOT EXISTS Mobile_Phone (
    ID INT NOT NULL AUTO_INCREMENT,
    Pkg_ID INT NOT NULL,
    Minutes_N INT NOT NULL,
    SMS_N INT NOT NULL,
    Minutes_Fee FLOAT NOT NULL CHECK (Minutes_Fee  >= 0),
    SMS_Fee FLOAT NOT NULL CHECK (SMS_Fee  >= 0),
    PRIMARY KEY ( ID ),
    FOREIGN KEY ( Pkg_ID )
        REFERENCES Service_Pkgs ( ID )
            ON UPDATE NO ACTION ON DELETE CASCADE
);

CREATE TABLE IF NOT EXISTS Internet (
    ID INT NOT NULL AUTO_INCREMENT,
    Pkg_ID INT NOT NULL,
    GB_N INT NOT NULL,
    GB_Fee FLOAT NOT NULL CHECK (GB_Fee  >= 0),
    Fixed BOOLEAN DEFAULT FALSE NOT NULL,
    PRIMARY KEY ( ID ),
    FOREIGN KEY ( Pkg_ID )
        REFERENCES Service_Pkgs ( ID )
            ON UPDATE NO ACTION ON DELETE CASCADE

);

CREATE TABLE IF NOT EXISTS Fixed_Phone (
    ID INT NOT NULL AUTO_INCREMENT,
    Pkg_ID INT NOT NULL,
    PRIMARY KEY ( ID ),
    FOREIGN KEY ( Pkg_ID )
        REFERENCES Service_Pkgs ( ID )
            ON UPDATE NO ACTION ON DELETE CASCADE
);

CREATE TABLE IF NOT EXISTS Optional_Products (
    ID INT NOT NULL AUTO_INCREMENT,
    Name VARCHAR(255) NOT NULL,
    Monthly_Fee FLOAT NOT NULL CHECK (Monthly_Fee  >= 0),
    PRIMARY KEY ( ID )
);

CREATE TABLE IF NOT EXISTS Validity_Periods (
    ID INT NOT NULL AUTO_INCREMENT,
    Pkg_ID INT NOT NULL,
    Months INT NOT NULL CHECK (Months  > 0),
    Monthly_Fee FLOAT NOT NULL CHECK (Monthly_Fee  >= 0),
    PRIMARY KEY ( ID ),
    FOREIGN KEY ( Pkg_ID )
        REFERENCES Service_Pkgs ( ID )
            ON UPDATE NO ACTION ON DELETE CASCADE
);


CREATE TABLE IF NOT EXISTS Orders (
    ID INT NOT NULL AUTO_INCREMENT,
    User_ID INT NOT NULL,
    Pkg_ID INT,
    Validity_Period_ID INT,
    Activation_Date DATE,
    Timestamp TIMESTAMP NOT NULL DEFAULT CURRENT_TIMESTAMP,
    Status ENUM('PENDING', 'FAILED', 'VALID') NOT NULL,
    Total FLOAT NOT NULL CHECK (Total  >= 0),
    PRIMARY KEY ( ID ),
    FOREIGN KEY ( User_ID )
        REFERENCES Users ( ID )
            ON UPDATE NO ACTION ON DELETE CASCADE,
    FOREIGN KEY ( Pkg_ID )
        REFERENCES Service_Pkgs ( ID )
            ON UPDATE NO ACTION ON DELETE SET NULL,
    FOREIGN KEY ( Validity_Period_ID )
        REFERENCES Validity_Periods ( ID )
            ON UPDATE NO ACTION ON DELETE SET NULL
);

CREATE TABLE IF NOT EXISTS OrderComprehendsOptional (
    Order_ID INT NOT NULL,
    Optional_ID INT NOT NULL,
    PRIMARY KEY ( Order_ID, Optional_ID ),
    FOREIGN KEY ( Order_ID )
        REFERENCES Orders ( ID )
            ON UPDATE NO ACTION ON DELETE CASCADE,
    FOREIGN KEY ( Optional_ID )
        REFERENCES Optional_Products ( ID )
            ON UPDATE NO ACTION ON DELETE CASCADE
);

CREATE TABLE IF NOT EXISTS ServicePkgOffersOptional (
    Pkg_ID INT NOT NULL,
    Optional_ID INT NOT NULL,
    PRIMARY KEY ( Pkg_ID, Optional_ID ),
    FOREIGN KEY ( Pkg_ID )
        REFERENCES Service_Pkgs ( ID )
            ON UPDATE NO ACTION ON DELETE CASCADE,
    FOREIGN KEY ( Optional_ID )
        REFERENCES Optional_Products ( ID )
            ON UPDATE NO ACTION ON DELETE CASCADE
);

CREATE TABLE IF NOT EXISTS Employees (
    ID INT NOT NULL AUTO_INCREMENT,
    Username varchar(255) UNIQUE NOT NULL,
    Password varchar(64) NOT NULL,
    PRIMARY KEY ( ID )
);

CREATE TABLE IF NOT EXISTS ServiceActivationSchedule (
    Order_ID INT NOT NULL,
    User_ID INT NOT NULL,
    Deactivation_Date DATE NOT NULL,
    PRIMARY KEY ( Order_ID, User_ID ),
    FOREIGN KEY ( Order_ID )
        REFERENCES Orders ( ID )
            ON UPDATE NO ACTION ON DELETE CASCADE,
    FOREIGN KEY ( User_ID )
        REFERENCES Users ( ID )
            ON UPDATE NO ACTION ON DELETE CASCADE
);


CREATE TABLE IF NOT EXISTS Audits (
    User_ID INT NOT NULL,
    Timestamp TIMESTAMP NOT NULL DEFAULT CURRENT_TIMESTAMP,
    Mail varchar(255) NOT NULL,
    Username varchar(255) NOT NULL,
    Amount FLOAT NOT NULL,
    PRIMARY KEY ( User_ID, Timestamp )
);
\end{lstlisting}



