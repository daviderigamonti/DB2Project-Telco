%---------------------------------------------------------------------
%   DATA MODELS CHAPTER
%---------------------------------------------------------------------

\chapter{Conceptual and logical data models}

\section{ER Diagram}

\begin{figure}[h]
    \centering
    \centerline{\includesvg[scale=0.45]{img/ER_diagram.svg}}
    \caption{ER diagram for the database}
    \label{fig:er_diagram}
\end{figure}

The proposed ER diagram reflects the directives indicated in the given specifications.

In some entities we deemed appropiate to add a generated incremental ID as the primary key even if there were other candidate keys available (e.g. the username of the \textit{User} could have been its primary key), this was done in order to simplify foreign key handling with the added benefit of being more maintainable in the long term.

Another doubt that arose during the planning phase of the database was related to the \textit{Fixed Phone} entity since, in the specifications, it was clearly stated that it shouldn't have any particular configuration parameters, therefore modeling it as an entity would mean having an "empty" entity; in the end we decided to stick with our choice for the sake of consistency with the other services and because it is also a scalable solution in case the company needs to add parameters to the service in the future.

The specifications document is not completely transparent regarding the \textit{Service Activation Schedule} and its purpose inside the database: although it is stated that the schedule must include the date of activation and deactivation for the services and the optional products for a given user, we opted to include just a reference to the user, a reference to the order and the deactivation date.

The previous choice was taken in order to reduce redundancy and minimize the overhead in terms of computation time and disk space for each \textit{Order} (in a real case scenario, ideally, the company would have a high number of orders) without compromising the available information, since we can retrieve all of the specified data from the order reference due to the fact that the services and optional products share the order's validity period.

\section{Logical model}
\todo{logical model}
\todo{NOTE: talk about ISA resolution}
\blindtext